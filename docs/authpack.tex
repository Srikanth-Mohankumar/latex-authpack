% Options for packages loaded elsewhere
\PassOptionsToPackage{unicode}{hyperref}
\PassOptionsToPackage{hyphens}{url}
%
\documentclass[
]{article}
\usepackage{lmodern}
\usepackage{amssymb,amsmath}
\usepackage{ifxetex,ifluatex}
\ifnum 0\ifxetex 1\fi\ifluatex 1\fi=0 % if pdftex
  \usepackage[T1]{fontenc}
  \usepackage[utf8]{inputenc}
  \usepackage{textcomp} % provide euro and other symbols
\else % if luatex or xetex
  \usepackage{unicode-math}
  \defaultfontfeatures{Scale=MatchLowercase}
  \defaultfontfeatures[\rmfamily]{Ligatures=TeX,Scale=1}
  \setmainfont[]{TeX Gyre Pagella}
  \setsansfont[]{TeX Gyre Heros}
  \setmonofont[]{Inconsolata}
\fi
% Use upquote if available, for straight quotes in verbatim environments
\IfFileExists{upquote.sty}{\usepackage{upquote}}{}
\IfFileExists{microtype.sty}{% use microtype if available
  \usepackage[]{microtype}
  \UseMicrotypeSet[protrusion]{basicmath} % disable protrusion for tt fonts
}{}
\makeatletter
\@ifundefined{KOMAClassName}{% if non-KOMA class
  \IfFileExists{parskip.sty}{%
    \usepackage{parskip}
  }{% else
    \setlength{\parindent}{0pt}
    \setlength{\parskip}{6pt plus 2pt minus 1pt}}
}{% if KOMA class
  \KOMAoptions{parskip=half}}
\makeatother
\usepackage{xcolor}
\IfFileExists{xurl.sty}{\usepackage{xurl}}{} % add URL line breaks if available
\IfFileExists{bookmark.sty}{\usepackage{bookmark}}{\usepackage{hyperref}}
\hypersetup{
  hidelinks,
  pdfcreator={LaTeX via pandoc}}
\urlstyle{same} % disable monospaced font for URLs
\usepackage[margin=1in]{geometry}
\usepackage{color}
\usepackage{fancyvrb}
\newcommand{\VerbBar}{|}
\newcommand{\VERB}{\Verb[commandchars=\\\{\}]}
\DefineVerbatimEnvironment{Highlighting}{Verbatim}{commandchars=\\\{\}}
% Add ',fontsize=\small' for more characters per line
\newenvironment{Shaded}{}{}
\newcommand{\AlertTok}[1]{\textcolor[rgb]{1.00,0.00,0.00}{\textbf{#1}}}
\newcommand{\AnnotationTok}[1]{\textcolor[rgb]{0.38,0.63,0.69}{\textbf{\textit{#1}}}}
\newcommand{\AttributeTok}[1]{\textcolor[rgb]{0.49,0.56,0.16}{#1}}
\newcommand{\BaseNTok}[1]{\textcolor[rgb]{0.25,0.63,0.44}{#1}}
\newcommand{\BuiltInTok}[1]{#1}
\newcommand{\CharTok}[1]{\textcolor[rgb]{0.25,0.44,0.63}{#1}}
\newcommand{\CommentTok}[1]{\textcolor[rgb]{0.38,0.63,0.69}{\textit{#1}}}
\newcommand{\CommentVarTok}[1]{\textcolor[rgb]{0.38,0.63,0.69}{\textbf{\textit{#1}}}}
\newcommand{\ConstantTok}[1]{\textcolor[rgb]{0.53,0.00,0.00}{#1}}
\newcommand{\ControlFlowTok}[1]{\textcolor[rgb]{0.00,0.44,0.13}{\textbf{#1}}}
\newcommand{\DataTypeTok}[1]{\textcolor[rgb]{0.56,0.13,0.00}{#1}}
\newcommand{\DecValTok}[1]{\textcolor[rgb]{0.25,0.63,0.44}{#1}}
\newcommand{\DocumentationTok}[1]{\textcolor[rgb]{0.73,0.13,0.13}{\textit{#1}}}
\newcommand{\ErrorTok}[1]{\textcolor[rgb]{1.00,0.00,0.00}{\textbf{#1}}}
\newcommand{\ExtensionTok}[1]{#1}
\newcommand{\FloatTok}[1]{\textcolor[rgb]{0.25,0.63,0.44}{#1}}
\newcommand{\FunctionTok}[1]{\textcolor[rgb]{0.02,0.16,0.49}{#1}}
\newcommand{\ImportTok}[1]{#1}
\newcommand{\InformationTok}[1]{\textcolor[rgb]{0.38,0.63,0.69}{\textbf{\textit{#1}}}}
\newcommand{\KeywordTok}[1]{\textcolor[rgb]{0.00,0.44,0.13}{\textbf{#1}}}
\newcommand{\NormalTok}[1]{#1}
\newcommand{\OperatorTok}[1]{\textcolor[rgb]{0.40,0.40,0.40}{#1}}
\newcommand{\OtherTok}[1]{\textcolor[rgb]{0.00,0.44,0.13}{#1}}
\newcommand{\PreprocessorTok}[1]{\textcolor[rgb]{0.74,0.48,0.00}{#1}}
\newcommand{\RegionMarkerTok}[1]{#1}
\newcommand{\SpecialCharTok}[1]{\textcolor[rgb]{0.25,0.44,0.63}{#1}}
\newcommand{\SpecialStringTok}[1]{\textcolor[rgb]{0.73,0.40,0.53}{#1}}
\newcommand{\StringTok}[1]{\textcolor[rgb]{0.25,0.44,0.63}{#1}}
\newcommand{\VariableTok}[1]{\textcolor[rgb]{0.10,0.09,0.49}{#1}}
\newcommand{\VerbatimStringTok}[1]{\textcolor[rgb]{0.25,0.44,0.63}{#1}}
\newcommand{\WarningTok}[1]{\textcolor[rgb]{0.38,0.63,0.69}{\textbf{\textit{#1}}}}
\usepackage{longtable,booktabs}
% Correct order of tables after \paragraph or \subparagraph
\usepackage{etoolbox}
\makeatletter
\patchcmd\longtable{\par}{\if@noskipsec\mbox{}\fi\par}{}{}
\makeatother
% Allow footnotes in longtable head/foot
\IfFileExists{footnotehyper.sty}{\usepackage{footnotehyper}}{\usepackage{footnote}}
\makesavenoteenv{longtable}
\setlength{\emergencystretch}{3em} % prevent overfull lines
\providecommand{\tightlist}{%
  \setlength{\itemsep}{0pt}\setlength{\parskip}{0pt}}
\setcounter{secnumdepth}{-\maxdimen} % remove section numbering

\author{}
\date{}

\begin{document}

\hypertarget{authpack---advanced-authoraffiliation-management-for-lualatex}{%
\section{authpack - Advanced Author/Affiliation Management for
LuaLaTeX}\label{authpack---advanced-authoraffiliation-management-for-lualatex}}

\textbf{Version:} 1.2\\
\textbf{Date:} 2025-10-16\\
\textbf{Author:} Srikanth Mohankumar\\
\textbf{License:} LaTeX Project Public License

\hypertarget{overview}{%
\subsection{Overview}\label{overview}}

\texttt{authpack} is a sophisticated LaTeX package for managing authors
and their affiliations in academic documents. It provides clickable
markers, ORCID integration, and flexible layout styles using Lua
scripting for maximum flexibility and performance.

\hypertarget{requirements}{%
\subsection{Requirements}\label{requirements}}

\begin{itemize}
\tightlist
\item
  \textbf{LuaLaTeX} (required - will not work with pdfLaTeX or XeLaTeX)
\item
  \texttt{hyperref} package (optional, for clickable links)
\item
  \texttt{graphicx} package (for ORCID icon display)
\item
  ORCID icon file: \texttt{orcid.pdf} (optional, for icon display)
\end{itemize}

\hypertarget{installation}{%
\subsection{Installation}\label{installation}}

\begin{enumerate}
\def\labelenumi{\arabic{enumi}.}
\tightlist
\item
  Place \texttt{authpack.sty} in your LaTeX project directory or in your
  local texmf tree
\item
  Optionally place \texttt{orcid.pdf} icon in the same directory
\item
  Compile your document with \texttt{lualatex}
\end{enumerate}

\hypertarget{basic-usage}{%
\subsection{Basic Usage}\label{basic-usage}}

\begin{Shaded}
\begin{Highlighting}[]
\BuiltInTok{\textbackslash{}documentclass}\NormalTok{\{}\ExtensionTok{article}\NormalTok{\}}
\BuiltInTok{\textbackslash{}usepackage}\NormalTok{\{}\ExtensionTok{hyperref}\NormalTok{\}}
\BuiltInTok{\textbackslash{}usepackage}\NormalTok{\{}\ExtensionTok{authpack}\NormalTok{\}}

\KeywordTok{\textbackslash{}begin}\NormalTok{\{}\ExtensionTok{document}\NormalTok{\}}

\CommentTok{\% Define affiliations}
\FunctionTok{\textbackslash{}Affil}\NormalTok{\{uni1\}\{University of Example\}}
\FunctionTok{\textbackslash{}Affil}\NormalTok{\{uni2\}\{Institute of Technology\}}

\CommentTok{\% Define authors}
\FunctionTok{\textbackslash{}Author}\NormalTok{\{author1\}\{John Doe\}\{affils=uni1, orcid=0000{-}0001{-}2345{-}6789\}}
\FunctionTok{\textbackslash{}Author}\NormalTok{\{author2\}\{Jane Smith\}\{affils=uni2\}}

\FunctionTok{\textbackslash{}title}\NormalTok{\{My Research Paper\}}
\FunctionTok{\textbackslash{}maketitle}

\KeywordTok{\textbackslash{}end}\NormalTok{\{}\ExtensionTok{document}\NormalTok{\}}
\end{Highlighting}
\end{Shaded}

\hypertarget{package-options}{%
\subsection{Package Options}\label{package-options}}

Load the package with options:

\begin{Shaded}
\begin{Highlighting}[]
\BuiltInTok{\textbackslash{}usepackage}\NormalTok{[style=inline, marker=num, orcid=icon, debug=false]\{}\ExtensionTok{authpack}\NormalTok{\}}
\end{Highlighting}
\end{Shaded}

\hypertarget{available-options}{%
\subsubsection{Available Options}\label{available-options}}

\begin{longtable}[]{@{}llll@{}}
\toprule
\begin{minipage}[b]{0.19\columnwidth}\raggedright
Option\strut
\end{minipage} & \begin{minipage}[b]{0.19\columnwidth}\raggedright
Values\strut
\end{minipage} & \begin{minipage}[b]{0.21\columnwidth}\raggedright
Default\strut
\end{minipage} & \begin{minipage}[b]{0.30\columnwidth}\raggedright
Description\strut
\end{minipage}\tabularnewline
\midrule
\endhead
\begin{minipage}[t]{0.19\columnwidth}\raggedright
\texttt{style}\strut
\end{minipage} & \begin{minipage}[t]{0.19\columnwidth}\raggedright
\texttt{inline}, \texttt{block}, \texttt{footnote}\strut
\end{minipage} & \begin{minipage}[t]{0.21\columnwidth}\raggedright
\texttt{inline}\strut
\end{minipage} & \begin{minipage}[t]{0.30\columnwidth}\raggedright
Layout style for authors/affiliations\strut
\end{minipage}\tabularnewline
\begin{minipage}[t]{0.19\columnwidth}\raggedright
\texttt{marker}\strut
\end{minipage} & \begin{minipage}[t]{0.19\columnwidth}\raggedright
\texttt{num}, \texttt{alpha}, \texttt{symbol}\strut
\end{minipage} & \begin{minipage}[t]{0.21\columnwidth}\raggedright
\texttt{num}\strut
\end{minipage} & \begin{minipage}[t]{0.30\columnwidth}\raggedright
Marker style for affiliations\strut
\end{minipage}\tabularnewline
\begin{minipage}[t]{0.19\columnwidth}\raggedright
\texttt{commasep}\strut
\end{minipage} & \begin{minipage}[t]{0.19\columnwidth}\raggedright
\texttt{true}, \texttt{false}\strut
\end{minipage} & \begin{minipage}[t]{0.21\columnwidth}\raggedright
\texttt{true}\strut
\end{minipage} & \begin{minipage}[t]{0.30\columnwidth}\raggedright
Use commas between author names\strut
\end{minipage}\tabularnewline
\begin{minipage}[t]{0.19\columnwidth}\raggedright
\texttt{oxford}\strut
\end{minipage} & \begin{minipage}[t]{0.19\columnwidth}\raggedright
\texttt{true}, \texttt{false}\strut
\end{minipage} & \begin{minipage}[t]{0.21\columnwidth}\raggedright
\texttt{true}\strut
\end{minipage} & \begin{minipage}[t]{0.30\columnwidth}\raggedright
Use Oxford comma before ``and''\strut
\end{minipage}\tabularnewline
\begin{minipage}[t]{0.19\columnwidth}\raggedright
\texttt{showand}\strut
\end{minipage} & \begin{minipage}[t]{0.19\columnwidth}\raggedright
\texttt{true}, \texttt{false}\strut
\end{minipage} & \begin{minipage}[t]{0.21\columnwidth}\raggedright
\texttt{true}\strut
\end{minipage} & \begin{minipage}[t]{0.30\columnwidth}\raggedright
Show ``and'' before last author\strut
\end{minipage}\tabularnewline
\begin{minipage}[t]{0.19\columnwidth}\raggedright
\texttt{orcid}\strut
\end{minipage} & \begin{minipage}[t]{0.19\columnwidth}\raggedright
\texttt{icon}, \texttt{text}, \texttt{none}\strut
\end{minipage} & \begin{minipage}[t]{0.21\columnwidth}\raggedright
\texttt{icon}\strut
\end{minipage} & \begin{minipage}[t]{0.30\columnwidth}\raggedright
ORCID display style\strut
\end{minipage}\tabularnewline
\begin{minipage}[t]{0.19\columnwidth}\raggedright
\texttt{orcidlink}\strut
\end{minipage} & \begin{minipage}[t]{0.19\columnwidth}\raggedright
\texttt{true}, \texttt{false}\strut
\end{minipage} & \begin{minipage}[t]{0.21\columnwidth}\raggedright
\texttt{true}\strut
\end{minipage} & \begin{minipage}[t]{0.30\columnwidth}\raggedright
Make ORCID clickable\strut
\end{minipage}\tabularnewline
\begin{minipage}[t]{0.19\columnwidth}\raggedright
\texttt{backlink}\strut
\end{minipage} & \begin{minipage}[t]{0.19\columnwidth}\raggedright
\texttt{true}, \texttt{false}\strut
\end{minipage} & \begin{minipage}[t]{0.21\columnwidth}\raggedright
\texttt{false}\strut
\end{minipage} & \begin{minipage}[t]{0.30\columnwidth}\raggedright
Add return links from affiliations (inline style only)\strut
\end{minipage}\tabularnewline
\begin{minipage}[t]{0.19\columnwidth}\raggedright
\texttt{debug}\strut
\end{minipage} & \begin{minipage}[t]{0.19\columnwidth}\raggedright
\texttt{true}, \texttt{false}\strut
\end{minipage} & \begin{minipage}[t]{0.21\columnwidth}\raggedright
\texttt{false}\strut
\end{minipage} & \begin{minipage}[t]{0.30\columnwidth}\raggedright
Enable debug output to console/log\strut
\end{minipage}\tabularnewline
\bottomrule
\end{longtable}

\hypertarget{commands}{%
\subsection{Commands}\label{commands}}

\hypertarget{affillabelidkeytext}{%
\subsubsection{\texorpdfstring{\texttt{\textbackslash{}Affil{[}label{]}{[}id{]}\{key\}\{text\}}}{\textbackslash Affil{[}label{]}{[}id{]}\{key\}\{text\}}}\label{affillabelidkeytext}}

Defines an affiliation.

\begin{itemize}
\tightlist
\item
  \texttt{label} (optional): Custom marker label (e.g., \texttt{*},
  \texttt{a}, \texttt{1})
\item
  \texttt{id} (optional): Custom HTML anchor ID for hyperlinks
\item
  \texttt{key} (required): Unique identifier for this affiliation
\item
  \texttt{text} (required): Affiliation text to display
\end{itemize}

\textbf{Examples:}

\begin{Shaded}
\begin{Highlighting}[]
\FunctionTok{\textbackslash{}Affil}\NormalTok{\{mit\}\{Massachusetts Institute of Technology\}}
\FunctionTok{\textbackslash{}Affil}\NormalTok{[*]\{harvard\}\{Harvard University\}}
\FunctionTok{\textbackslash{}Affil}\NormalTok{[A][custom{-}id]\{oxford\}\{University of Oxford\}}
\end{Highlighting}
\end{Shaded}

\hypertarget{authorkeynameoptions}{%
\subsubsection{\texorpdfstring{\texttt{\textbackslash{}Author\{key\}\{name\}\{options\}}}{\textbackslash Author\{key\}\{name\}\{options\}}}\label{authorkeynameoptions}}

Defines an author with optional metadata.

\begin{itemize}
\tightlist
\item
  \texttt{key} (required): Unique identifier for this author
\item
  \texttt{name} (required): Author's full name
\item
  \texttt{options} (required): Comma-separated key=value pairs
\end{itemize}

\textbf{Options:}

\begin{itemize}
\tightlist
\item
  \texttt{affils}: Comma-separated list of affiliation keys (supports
  multiple affiliations)
\item
  \texttt{email}: Email address
\item
  \texttt{orcid}: ORCID identifier (format: 0000-0000-0000-0000)
\item
  \texttt{marker}: Custom marker (rarely needed)
\end{itemize}

\textbf{Examples:}

\begin{Shaded}
\begin{Highlighting}[]
\FunctionTok{\textbackslash{}Author}\NormalTok{\{jdoe\}\{John Doe\}\{affils=mit, orcid=0000{-}0001{-}2345{-}6789\}}
\FunctionTok{\textbackslash{}Author}\NormalTok{\{jsmith\}\{Jane Smith\}\{affils=mit,harvard, email=jane@example.com\}}
\FunctionTok{\textbackslash{}Author}\NormalTok{\{bob\}\{Bob Johnson\}\{affils=oxford\}}
\end{Highlighting}
\end{Shaded}

\hypertarget{printauthors}{%
\subsubsection{\texorpdfstring{\texttt{\textbackslash{}PrintAuthors}}{\textbackslash PrintAuthors}}\label{printauthors}}

Manually renders the author and affiliation block. Usually not needed as
\texttt{\textbackslash{}maketitle} automatically calls this.

\hypertarget{styles}{%
\subsection{Styles}\label{styles}}

\hypertarget{inline-style-default}{%
\subsubsection{Inline Style (default)}\label{inline-style-default}}

Authors listed in a single paragraph with superscript affiliation
markers, followed by affiliation list.

\begin{Shaded}
\begin{Highlighting}[]
\BuiltInTok{\textbackslash{}usepackage}\NormalTok{[style=inline]\{}\ExtensionTok{authpack}\NormalTok{\}}
\end{Highlighting}
\end{Shaded}

\textbf{Output format:}

\begin{verbatim}
John Doe¹, Jane Smith¹,² and Bob Johnson³

¹Massachusetts Institute of Technology
²Harvard University
³University of Oxford
\end{verbatim}

\textbf{Features:} - Authors in single line with clickable superscript
markers - Affiliations listed below with hyperlink targets - Optional
backlinks from affiliations to author list - Handles multiple
affiliations per author

\hypertarget{block-style}{%
\subsubsection{Block Style}\label{block-style}}

Authors grouped by their affiliation combinations with affiliation text
below each group.

\begin{Shaded}
\begin{Highlighting}[]
\BuiltInTok{\textbackslash{}usepackage}\NormalTok{[style=block]\{}\ExtensionTok{authpack}\NormalTok{\}}
\end{Highlighting}
\end{Shaded}

\textbf{Output format:}

\begin{verbatim}
John Doe¹ and Jane Smith¹,²
¹Massachusetts Institute of Technology
²Harvard University

Bob Johnson³
³University of Oxford
\end{verbatim}

\textbf{Features:} - Authors grouped by shared affiliations - Each group
shows relevant affiliations immediately below - Ideal for documents with
clear institutional groupings

\hypertarget{footnote-style}{%
\subsubsection{Footnote Style}\label{footnote-style}}

Authors listed with superscript markers, affiliations rendered as LaTeX
footnotes at page bottom.

\begin{Shaded}
\begin{Highlighting}[]
\BuiltInTok{\textbackslash{}usepackage}\NormalTok{[style=footnote]\{}\ExtensionTok{authpack}\NormalTok{\}}
\end{Highlighting}
\end{Shaded}

\textbf{Output format:}

\begin{verbatim}
John Doe¹, Jane Smith¹,² and Bob Johnson³
\end{verbatim}

With footnotes at bottom of page: - ¹ Massachusetts Institute of
Technology - ² Harvard University - ³ University of Oxford

\textbf{Features:} - Uses \texttt{\textbackslash{}footnotetext} for
proper footnote placement - Clickable superscript markers link to
footnotes - Each affiliation appears exactly once - Ideal for
traditional academic paper format

\hypertarget{marker-styles}{%
\subsection{Marker Styles}\label{marker-styles}}

\hypertarget{numeric-default}{%
\subsubsection{Numeric (default)}\label{numeric-default}}

\begin{Shaded}
\begin{Highlighting}[]
\BuiltInTok{\textbackslash{}usepackage}\NormalTok{[marker=num]\{}\ExtensionTok{authpack}\NormalTok{\}}
\end{Highlighting}
\end{Shaded}

Produces: ¹, ², ³, 4, 5\ldots{}

\hypertarget{alphabetic}{%
\subsubsection{Alphabetic}\label{alphabetic}}

\begin{Shaded}
\begin{Highlighting}[]
\BuiltInTok{\textbackslash{}usepackage}\NormalTok{[marker=alpha]\{}\ExtensionTok{authpack}\NormalTok{\}}
\end{Highlighting}
\end{Shaded}

Produces: A, B, C, D, E\ldots{}

\hypertarget{symbolic}{%
\subsubsection{Symbolic}\label{symbolic}}

\begin{Shaded}
\begin{Highlighting}[]
\BuiltInTok{\textbackslash{}usepackage}\NormalTok{[marker=symbol]\{}\ExtensionTok{authpack}\NormalTok{\}}
\end{Highlighting}
\end{Shaded}

Produces: *, †, ‡, §, ¶, \textbar\textbar, **, ††, ‡‡

\hypertarget{orcid-integration}{%
\subsection{ORCID Integration}\label{orcid-integration}}

\hypertarget{icon-display-default}{%
\subsubsection{Icon Display (default)}\label{icon-display-default}}

\begin{Shaded}
\begin{Highlighting}[]
\BuiltInTok{\textbackslash{}usepackage}\NormalTok{[orcid=icon]\{}\ExtensionTok{authpack}\NormalTok{\}}
\end{Highlighting}
\end{Shaded}

Displays the ORCID icon next to author names (requires
\texttt{orcid.pdf}).

\hypertarget{text-display}{%
\subsubsection{Text Display}\label{text-display}}

\begin{Shaded}
\begin{Highlighting}[]
\BuiltInTok{\textbackslash{}usepackage}\NormalTok{[orcid=text]\{}\ExtensionTok{authpack}\NormalTok{\}}
\end{Highlighting}
\end{Shaded}

Displays ``ORCID: 0000-0000-0000-0000'' next to author names.

\hypertarget{no-display}{%
\subsubsection{No Display}\label{no-display}}

\begin{Shaded}
\begin{Highlighting}[]
\BuiltInTok{\textbackslash{}usepackage}\NormalTok{[orcid=none]\{}\ExtensionTok{authpack}\NormalTok{\}}
\end{Highlighting}
\end{Shaded}

Hides ORCID information.

\hypertarget{clickable-orcid-links}{%
\subsubsection{Clickable ORCID Links}\label{clickable-orcid-links}}

By default, ORCID identifiers are clickable and link to
\texttt{https://orcid.org/{[}ID{]}}. Disable with:

\begin{Shaded}
\begin{Highlighting}[]
\BuiltInTok{\textbackslash{}usepackage}\NormalTok{[orcidlink=false]\{}\ExtensionTok{authpack}\NormalTok{\}}
\end{Highlighting}
\end{Shaded}

\hypertarget{advanced-features}{%
\subsection{Advanced Features}\label{advanced-features}}

\hypertarget{multiple-affiliations-per-author}{%
\subsubsection{Multiple Affiliations per
Author}\label{multiple-affiliations-per-author}}

Authors can have multiple affiliations with proper marker display:

\begin{Shaded}
\begin{Highlighting}[]
\FunctionTok{\textbackslash{}Affil}\NormalTok{\{uni1\}\{University A\}}
\FunctionTok{\textbackslash{}Affil}\NormalTok{\{uni2\}\{University B\}}
\FunctionTok{\textbackslash{}Affil}\NormalTok{\{uni3\}\{University C\}}

\FunctionTok{\textbackslash{}Author}\NormalTok{\{jdoe\}\{John Doe\}\{affils=uni1,uni2,uni3\}}
\CommentTok{\% Displays as: John Doe¹,²,³}
\end{Highlighting}
\end{Shaded}

\textbf{Important:} No spaces in the affiliation list value: - ✓
Correct: \texttt{affils=uni1,uni2,uni3} - ✗ Wrong:
\texttt{affils=uni1,\ uni2,\ uni3} (spaces will cause parsing issues)

\hypertarget{custom-markers}{%
\subsubsection{Custom Markers}\label{custom-markers}}

Override automatic marker generation:

\begin{Shaded}
\begin{Highlighting}[]
\FunctionTok{\textbackslash{}Affil}\NormalTok{[*]\{primary\}\{Primary Institution\}}
\FunctionTok{\textbackslash{}Affil}\NormalTok{[†]\{secondary\}\{Secondary Institution\}}
\end{Highlighting}
\end{Shaded}

\hypertarget{clickable-links}{%
\subsubsection{Clickable Links}\label{clickable-links}}

When \texttt{hyperref} is loaded, affiliation markers are clickable and
link to the affiliation text.

\textbf{Inline style:} Use \texttt{backlink=true} to add return links
from affiliations:

\begin{Shaded}
\begin{Highlighting}[]
\BuiltInTok{\textbackslash{}usepackage}\NormalTok{[style=inline, backlink=true]\{}\ExtensionTok{authpack}\NormalTok{\}}
\end{Highlighting}
\end{Shaded}

\textbf{Footnote style:} Superscript markers are automatically clickable
and link to footnotes at page bottom.

\hypertarget{duplicate-label-handling}{%
\subsubsection{Duplicate Label
Handling}\label{duplicate-label-handling}}

If duplicate markers are detected, the package automatically adds
suffixes (a, b, c, etc.) and issues a warning.

\hypertarget{debug-mode}{%
\subsubsection{Debug Mode}\label{debug-mode}}

Enable comprehensive debug output to trace package behavior:

\begin{Shaded}
\begin{Highlighting}[]
\BuiltInTok{\textbackslash{}usepackage}\NormalTok{[debug=true]\{}\ExtensionTok{authpack}\NormalTok{\}}
\end{Highlighting}
\end{Shaded}

Debug output includes: - Affiliation registration with keys and markers
- Author registration with affiliation lists - Marker generation and
assignment - Complete data dump before rendering - Step-by-step
rendering process

Debug messages appear in console output and \texttt{.log} file.

\hypertarget{complete-examples}{%
\subsection{Complete Examples}\label{complete-examples}}

\hypertarget{example-1-basic-multi-affiliation}{%
\subsubsection{Example 1: Basic
Multi-Affiliation}\label{example-1-basic-multi-affiliation}}

\begin{Shaded}
\begin{Highlighting}[]
\BuiltInTok{\textbackslash{}documentclass}\NormalTok{\{}\ExtensionTok{article}\NormalTok{\}}
\BuiltInTok{\textbackslash{}usepackage}\NormalTok{\{}\ExtensionTok{hyperref}\NormalTok{\}}
\BuiltInTok{\textbackslash{}usepackage}\NormalTok{[style=inline]\{}\ExtensionTok{authpack}\NormalTok{\}}

\KeywordTok{\textbackslash{}begin}\NormalTok{\{}\ExtensionTok{document}\NormalTok{\}}

\FunctionTok{\textbackslash{}Affil}\NormalTok{\{mit\}\{Massachusetts Institute of Technology\}}
\FunctionTok{\textbackslash{}Affil}\NormalTok{\{harvard\}\{Harvard University\}}
\FunctionTok{\textbackslash{}Affil}\NormalTok{\{oxford\}\{University of Oxford\}}

\FunctionTok{\textbackslash{}Author}\NormalTok{\{jdoe\}\{John Doe\}\{affils=mit, orcid=0000{-}0001{-}2345{-}6789\}}
\FunctionTok{\textbackslash{}Author}\NormalTok{\{jsmith\}\{Jane Smith\}\{affils=mit,harvard\}}
\FunctionTok{\textbackslash{}Author}\NormalTok{\{bjohnson\}\{Bob Johnson\}\{affils=oxford\}}

\FunctionTok{\textbackslash{}title}\NormalTok{\{My Research Paper\}}
\FunctionTok{\textbackslash{}maketitle}

\KeywordTok{\textbackslash{}end}\NormalTok{\{}\ExtensionTok{document}\NormalTok{\}}
\end{Highlighting}
\end{Shaded}

\hypertarget{example-2-block-style-with-custom-markers}{%
\subsubsection{Example 2: Block Style with Custom
Markers}\label{example-2-block-style-with-custom-markers}}

\begin{Shaded}
\begin{Highlighting}[]
\BuiltInTok{\textbackslash{}documentclass}\NormalTok{\{}\ExtensionTok{article}\NormalTok{\}}
\BuiltInTok{\textbackslash{}usepackage}\NormalTok{\{}\ExtensionTok{hyperref}\NormalTok{\}}
\BuiltInTok{\textbackslash{}usepackage}\NormalTok{[style=block, marker=alpha]\{}\ExtensionTok{authpack}\NormalTok{\}}

\KeywordTok{\textbackslash{}begin}\NormalTok{\{}\ExtensionTok{document}\NormalTok{\}}

\FunctionTok{\textbackslash{}Affil}\NormalTok{\{inst1\}\{First Institution\}}
\FunctionTok{\textbackslash{}Affil}\NormalTok{\{inst2\}\{Second Institution\}}

\FunctionTok{\textbackslash{}Author}\NormalTok{\{auth1\}\{Author One\}\{affils=inst1,inst2\}}
\FunctionTok{\textbackslash{}Author}\NormalTok{\{auth2\}\{Author Two\}\{affils=inst2\}}

\FunctionTok{\textbackslash{}title}\NormalTok{\{Collaborative Research\}}
\FunctionTok{\textbackslash{}maketitle}

\KeywordTok{\textbackslash{}end}\NormalTok{\{}\ExtensionTok{document}\NormalTok{\}}
\end{Highlighting}
\end{Shaded}

\hypertarget{example-3-footnote-style-for-traditional-papers}{%
\subsubsection{Example 3: Footnote Style for Traditional
Papers}\label{example-3-footnote-style-for-traditional-papers}}

\begin{Shaded}
\begin{Highlighting}[]
\BuiltInTok{\textbackslash{}documentclass}\NormalTok{\{}\ExtensionTok{article}\NormalTok{\}}
\BuiltInTok{\textbackslash{}usepackage}\NormalTok{\{}\ExtensionTok{hyperref}\NormalTok{\}}
\BuiltInTok{\textbackslash{}usepackage}\NormalTok{[style=footnote, marker=num]\{}\ExtensionTok{authpack}\NormalTok{\}}

\KeywordTok{\textbackslash{}begin}\NormalTok{\{}\ExtensionTok{document}\NormalTok{\}}

\FunctionTok{\textbackslash{}Affil}\NormalTok{\{univ\}\{University Name\}}
\FunctionTok{\textbackslash{}Affil}\NormalTok{\{lab\}\{Laboratory Name\}}

\FunctionTok{\textbackslash{}Author}\NormalTok{\{lead\}\{Lead Author\}\{affils=univ,lab, orcid=0000{-}0001{-}2345{-}6789\}}
\FunctionTok{\textbackslash{}Author}\NormalTok{\{second\}\{Second Author\}\{affils=univ\}}

\FunctionTok{\textbackslash{}title}\NormalTok{\{Traditional Academic Paper\}}
\FunctionTok{\textbackslash{}maketitle}

\KeywordTok{\textbackslash{}end}\NormalTok{\{}\ExtensionTok{document}\NormalTok{\}}
\end{Highlighting}
\end{Shaded}

\hypertarget{troubleshooting}{%
\subsection{Troubleshooting}\label{troubleshooting}}

\hypertarget{luatex-required-error}{%
\subsubsection{``LuaTeX required'' error}\label{luatex-required-error}}

\textbf{Solution:} Compile with \texttt{lualatex} instead of
\texttt{pdflatex}:

\begin{Shaded}
\begin{Highlighting}[]
\ExtensionTok{lualatex}\NormalTok{ mydocument.tex}
\end{Highlighting}
\end{Shaded}

\hypertarget{orcid-icon-not-displaying}{%
\subsubsection{ORCID icon not
displaying}\label{orcid-icon-not-displaying}}

\textbf{Solution:} 1. Download the ORCID icon from
https://orcid.org/trademark-and-id-display-guidelines 2. Save as
\texttt{orcid.pdf} in your document directory 3. Or use
\texttt{orcid=text} option instead

\hypertarget{markers-not-clickable}{%
\subsubsection{Markers not clickable}\label{markers-not-clickable}}

\textbf{Solution:} Load \texttt{hyperref} package before
\texttt{authpack}:

\begin{Shaded}
\begin{Highlighting}[]
\BuiltInTok{\textbackslash{}usepackage}\NormalTok{\{}\ExtensionTok{hyperref}\NormalTok{\}}
\BuiltInTok{\textbackslash{}usepackage}\NormalTok{\{}\ExtensionTok{authpack}\NormalTok{\}}
\end{Highlighting}
\end{Shaded}

\hypertarget{multiple-affiliations-not-showing-all-markers}{%
\subsubsection{Multiple affiliations not showing all
markers}\label{multiple-affiliations-not-showing-all-markers}}

\textbf{Problem:} Only first affiliation appears for authors with
multiple affiliations.

\textbf{Solution:} Ensure no spaces in affiliation list: - ✓ Use:
\texttt{affils=mit,harvard,oxford} - ✗ Don't use:
\texttt{affils=mit,\ harvard,\ oxford}

\hypertarget{duplicate-label-warnings}{%
\subsubsection{Duplicate label
warnings}\label{duplicate-label-warnings}}

\textbf{Solution:} Use explicit labels with
\texttt{\textbackslash{}Affil{[}label{]}} to control marker assignment.

\hypertarget{footnotes-not-appearing-at-page-bottom}{%
\subsubsection{Footnotes not appearing at page
bottom}\label{footnotes-not-appearing-at-page-bottom}}

\textbf{Solution:} Ensure you're using \texttt{style=footnote}. The
package uses \texttt{\textbackslash{}footnotetext} which requires proper
page layout. If footnotes still don't appear, check for conflicting
packages.

\hypertarget{debug-output-needed}{%
\subsubsection{Debug output needed}\label{debug-output-needed}}

\textbf{Solution:} Enable debug mode to see detailed processing:

\begin{Shaded}
\begin{Highlighting}[]
\BuiltInTok{\textbackslash{}usepackage}\NormalTok{[debug=true]\{}\ExtensionTok{authpack}\NormalTok{\}}
\end{Highlighting}
\end{Shaded}

Check console output and \texttt{.log} file for debug messages.

\hypertarget{limitations}{%
\subsection{Limitations}\label{limitations}}

\begin{itemize}
\tightlist
\item
  Requires LuaLaTeX (not compatible with pdfLaTeX or XeLaTeX)
\item
  ORCID icon requires external \texttt{orcid.pdf} file
\item
  Maximum of \textasciitilde81 symbols in symbol mode (9 base symbols ×
  9 repetitions)
\item
  Affiliation lists in \texttt{\textbackslash{}Author} command must not
  contain spaces around commas
\item
  \texttt{backlink} option only works with \texttt{inline} style
\end{itemize}

\hypertarget{version-history}{%
\subsection{Version History}\label{version-history}}

\hypertarget{version-1.2-2025-10-16}{%
\subsubsection{Version 1.2 (2025-10-16)}\label{version-1.2-2025-10-16}}

\begin{itemize}
\tightlist
\item
  \textbf{Fixed:} Multi-affiliation support now works correctly in all
  styles
\item
  \textbf{Fixed:} Key-value parser now handles comma-separated
  affiliation lists properly
\item
  \textbf{Fixed:} Footnote style now uses
  \texttt{\textbackslash{}footnotetext} with proper placement
\item
  \textbf{Added:} Clickable hyperlinks in footnote style markers
\item
  \textbf{Added:} \texttt{debug} option for comprehensive
  troubleshooting
\item
  \textbf{Added:} Affiliation order tracking for consistent output
\item
  \textbf{Improved:} Block style now groups authors by affiliation
  combinations
\item
  \textbf{Improved:} Better handling of multiple affiliations per author
\end{itemize}

\hypertarget{version-1.0-2025-10-15}{%
\subsubsection{Version 1.0 (2025-10-15)}\label{version-1.0-2025-10-15}}

\begin{itemize}
\tightlist
\item
  Initial release
\item
  Inline and block styles
\item
  Multiple marker types (numeric, alphabetic, symbolic)
\item
  ORCID integration with clickable links
\item
  Automatic duplicate label handling
\item
  Hyperlink support with optional backlinks
\end{itemize}

\hypertarget{contributing}{%
\subsection{Contributing}\label{contributing}}

Report bugs and request features at the package repository.

\hypertarget{license}{%
\subsection{License}\label{license}}

This package is released under the LaTeX Project Public License v1.3c or
later.

\hypertarget{author}{%
\subsection{Author}\label{author}}

Srikanth Mohankumar

For support, please use the issue tracker or consult the debug output
with \texttt{debug=true} enabled.

\end{document}
